% LaTeX2e Template by Stephen Iota (https://stepheniota.github.io/)
% last updated: Aug. 2018

% for papers
%\documentclass[aps,onecolumn,superscriptaddress]{revtex4-1}

% https://www-d0.fnal.gov/Run2Physics/WWW/templates/revtex4.pdf
% https://cdn.journals.aps.org/files/revtex/auguide4-1.pdf
% for revTeX4-1 class options

% for other
\documentclass[11pt]{article}
\usepackage[margin=2cm]{geometry}

%%%%%%%%%%%%%%%%
%%% Packages %%%
%%%%%%%%%%%%%%%%

\usepackage[utf8]{inputenc}
\usepackage{amsmath}
\usepackage{amssymb}
\usepackage{amsfonts}
\usepackage{graphicx}
\usepackage[dvipsnames]{xcolor} % for colored links

%\usepackage{pdfpages}
%\makeatletter
%\AtBeginDocument{\let\LS@rot\@undefined}
%\makeatother


% always put this at the end
\usepackage[
	colorlinks=true,
	citecolor=green!50!black,
	linkcolor=NavyBlue!75!black,
	urlcolor=green!50!black,
	hypertexnames=false]{hyperref} 

 
%%%%%%%%%%%%%%%%
%%% Commands %%%
%%%%%%%%%%%%%%%%


%%%%%%%%%%%%%%%%%%
%% Front Matter %%
%%%%%%%%%%%%%%%%%%

% no page numbers
\pagenumbering{gobble}


%%%%%%%%%%%%%
%%% Title %%%
%%%%%%%%%%%%%
\begin{document}

\begin{center}
\Large\textbf{{Physics 40B Spring 2019}}

\large{\textsc{Supplemental Instruction Syllabus}} 


\end{center}




%%%%%%%%%%%%%%%%%%%%
%%%  Logistics   %%%
%%%%%%%%%%%%%%%%%%%%

\section*{Logistics}

\begin{tabular}{rl}

\textsc{SI Leader}:
&
Stephen Iota
\\
\textsc{Contact}:
&
\href{mailto:siota001@ucr.edu}{\texttt{siota001@ucr.edu}}
\\
\textsc{Webpage}:
&
\url{https://stepheniota.com/physics-40b-s19}
\\
\textsc{SI Sessions}:
&
\textbf{TWR 2 -- 3 pm} in \textbf{Skye 110}
\\
\textsc{Lecture:}
&
Dr.~Barsukov; TR 12:40 -- 2:00 pm  in Physics 2000
\\
\end{tabular}





%%%%%%%%%%%%%%%%%%%%%%%%%%
%% Official Description %%
%%%%%%%%%%%%%%%%%%%%%%%%%%


\section*{What is Supplemental Instruction?}

Supplemental Instruction \textsc{(SI)} is a free (!)\ academic support program that is designed to help students succeed in traditionally difficult courses. 
Taught by a peer mentor, SI focuses on reviewing material in addition to lectures and discussions, combining ``what to learn'' with ``how to learn.''
It has been statistically shown that students who consistently participate in SI sessions receive \textbf{higher course grades and have higher graduation rates} vs.~those who never attend.\footnote{Erin M.~Buchanan, Kathrene D.~Valentine \& Michael L.~Frizell (2018) Supplemental Instruction: Understanding Academic Assistance in Underrepresented Groups, The Journal of Experimental Education, DOI: 10.1080/00220973.2017.1421517}

\section*{Session Goals}
Physics 40B is probably on of the first ``applied'' math courses many of you will take. This class serves as the first stepping stone in a STEM career, where you will learn to use math to describe the world around you.
The main goal of my SI sessions is to have you start to \emph{think like a scientist or engineer}. We will focus on critical thinking skills and problem solving techniques that will prove invaluable throughout your careers in STEM.
Additionally, I will emphasize `soft skills' such as public speaking and science communication. These skills are sometimes overlooked, but can be crucial in excelling in internships and landing top jobs.

\section*{Grades \& Participation}

SI is not a graded course. However, as members of a \textsc{BCOE} Learning Community, you are required to attend SI sessions. I can assure you that if you participate, you will get something useful out of SI!

\section*{Guidelines}

Please respect one another and the classroom guidelines.  
This means: [1] always sign in before sessions, [2] no food or drink in classrooms, [3] no cell phone use during sessions and [4] no foul language or abusive behavior.
No hate speech will be tolerated and those guilty will be reported to the SI supervisor immediately. 

\section*{Final Remarks}

I look forward to a great quarter with all of you! My main purpose here as an SI is to be a resource for you. Please don't hesitate to contact me with any questions you may have, be them physics related or not!





\end{document}