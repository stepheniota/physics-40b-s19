%% LaTeX2e Template by Stephen Iota (https://stepheniota.com/)
%% last updated: Feb. 2019
%% for papers
%\documentclass[aps,preprint,notitlepage]{revtex4-1}
%% https://www-d0.fnal.gov/Run2Physics/WWW/templates/revtex4.pdf
%% https://cdn.journals.aps.org/files/revtex/auguide4-1.pdf
%% ^^ revTeX4-1 class options

%% for other
\documentclass[11pt]{article}
\usepackage{geometry} % let's be honest, standard LaTeX margins are GaRbagE for most purposes
%%%%%%%%%%%%%%%%
%%% Packages %%%
%%%%%%%%%%%%%%%%

\usepackage[utf8]{inputenc}
\usepackage[noadjust]{cite}
\usepackage{lipsum}
\usepackage{amsmath}
\usepackage{amssymb}
\usepackage{amsfonts}
\usepackage{physics} %http://ftp.math.purdue.edu/mirrors/ctan.org/macros/latex/contrib/physics/physics.pdf
\usepackage[thinc]{esdiff} % easy derivatives
\usepackage{graphicx} % \includegraphics{ }
\usepackage[shortlabels]{enumitem} % change labels in enum/item environments
\usepackage[dvipsnames]{xcolor} % colored links=
%\usepackage{footmisc} % http://mirror.utexas.edu/ctan/macros/latex/contrib/footmisc/footmisc.pdf
%\usepackage[small]{titlesec} % [small,medium,big] << controls size of *section text
%\usepackage{fancyhdr} %http://tug.ctan.org/tex-archive/macros/latex/contrib/fancyhdr/fancyhdr.pdf
% always put this at the end
\usepackage[
	colorlinks=true,
	citecolor=NavyBlue!90!black,
	linkcolor=NavyBlue!75!black,
	urlcolor=green!50!black,
	hypertexnames=false]{hyperref}

 %%%%%%%%%%%%%%%%%%
 %% New Commands %%
 %%%%%%%%%%%%%%%%%%
\newcommand{\email}[1]{\texttt{\href{mailto:#1}{#1}}}


%%%%%%%%%%%%%%%%%%
%% Front Matter %%
%%%%%%%%%%%%%%%%%%

%\pagenumbering{gobble} % no page numbers
%\graphicspath{{figures/}} % set directory for figures
\setcounter{section}{-1} % start with section 0


%%%%%%%%%%%%%
%%% Title %%%
%%%%%%%%%%%%%
\begin{document}

\begin{center}

\Large{\textsc{PSet 1}: \textbf{Theory of Gravitation}}
\end{center}
\vspace{.5mm}


%%%%%%%%%%
%% INFO %%
%%%%%%%%%%

\begin{tabular}{rl}
\textsc{SI Leader}:
&
Stephen Iota (\email{siota001@ucr.edu})
\\
\textsc{Course}:
&
Physics 40B (Spring 2019), Prof.~Barsukov
\\
\textsc{Date}:
&
\today
\end{tabular}

%%%%%%%%%%%%%%
%% PROBLEMS %%
%%%%%%%%%%%%%%


\section{Conceptual Questions}

\begin{enumerate}[(i)]
	\item Explain \textit{why} the gravitational potential energy between two masses is negative. Note: saying ``because that's what the equation gives'' is \textit{not} a valid answer. 
	\item The escape speed from Planet X is 10,000 m/s. Planet Y has the same radius as Planet X, but is twice as dense. What is the escape speed from Planet Y?
	\item Explain what is a geosynchronous orbit is, and explain what are the necessary conditions for such an orbit. 
\end{enumerate}

\section{Newton's Law of Gravitation}

A 20 kg sphere is at the origin and a 10 kg sphere is at x = 20 cm. At what position on the $x$-axis could you place a small mass such that the net gravitational force on it due to the spheres is zero?
 
 
 
 \section{Geosynchronous orbits on Mars}
 What are the speed and height of a geosynchronous satellite orbiting Mars? Mars rotates about its axis every 24.8 hours. 


\section{Orbital Energetics}

Show that, for a satellite in a circular orbit, $K = -\frac{1}{2} U_G$.


\section{Raising a satellite}
How much work must be done to boost a 1000 kg communications satellite from a low earth orbit with $h = 300$ km to a geosynchronous orbit?



\section{Schwarzschild radius}

Nothing can escape the \textit{event horizon} of a black hole, not even light. You can think of the event horizon as being the distance from a black hole at which the escape speed is the speed of light $c = 3\text{e}8$ m/s, making all escapes impossible. What is the radius of the event horizon for a black hole with a mass 5.0 times the mass of the sun? This distance is called the \textit{Schwarzschild radius}.





\end{document}
